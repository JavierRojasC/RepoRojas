%  LaTeX support: latex@mdpi.com
%  In case you need support, please attach all files that are necessary for compiling as well as the log file, and specify the details of your LaTeX setup (which operating system and LaTeX version / tools you are using).

%=================================================================
\documentclass[water,article,submit,moreauthors,pdftex]{mdpi}

% If you would like to post an early version of this manuscript as a preprint, you may use preprint as the journal and change 'submit' to 'accept'. The document class line would be, e.g., \documentclass[preprints,article,accept,moreauthors,pdftex]{mdpi}. This is especially recommended for submission to arXiv, where line numbers should be removed before posting. For preprints.org, the editorial staff will make this change immediately prior to posting.

%% Some pieces required from the pandoc template
\providecommand{\tightlist}{%
  \setlength{\itemsep}{0pt}\setlength{\parskip}{4pt}}
\setlist[itemize]{leftmargin=*,labelsep=5.8mm}
\setlist[enumerate]{leftmargin=*,labelsep=4.9mm}

\usepackage{longtable}

% see https://stackoverflow.com/a/47122900

%--------------------
% Class Options:
%--------------------
%----------
% journal
%----------
% Choose between the following MDPI journals:
% acoustics, actuators, addictions, admsci, aerospace, agriculture, agriengineering, agronomy, algorithms, animals, antibiotics, antibodies, antioxidants, applsci, arts, asc, asi, atmosphere, atoms, axioms, batteries, bdcc, behavsci , beverages, bioengineering, biology, biomedicines, biomimetics, biomolecules, biosensors, brainsci , buildings, cancers, carbon , catalysts, cells, ceramics, challenges, chemengineering, chemistry, chemosensors, children, cleantechnol, climate, clockssleep, cmd, coatings, colloids, computation, computers, condensedmatter, cosmetics, cryptography, crystals, dairy, data, dentistry, designs , diagnostics, diseases, diversity, drones, econometrics, economies, education, electrochem, electronics, energies, entropy, environments, epigenomes, est, fermentation, fibers, fire, fishes, fluids, foods, forecasting, forests, fractalfract, futureinternet, futurephys, galaxies, games, gastrointestdisord, gels, genealogy, genes, geohazards, geosciences, geriatrics, hazardousmatters, healthcare, heritage, highthroughput, horticulturae, humanities, hydrology, ijerph, ijfs, ijgi, ijms, ijns, ijtpp, informatics, information, infrastructures, inorganics, insects, instruments, inventions, iot, j, jcdd, jcm, jcp, jcs, jdb, jfb, jfmk, jimaging, jintelligence, jlpea, jmmp, jmse, jnt, jof, joitmc, jpm, jrfm, jsan, land, languages, laws, life, literature, logistics, lubricants, machines, magnetochemistry, make, marinedrugs, materials, mathematics, mca, medicina, medicines, medsci, membranes, metabolites, metals, microarrays, micromachines, microorganisms, minerals, modelling, molbank, molecules, mps, mti, nanomaterials, ncrna, neuroglia, nitrogen, notspecified, nutrients, ohbm, particles, pathogens, pharmaceuticals, pharmaceutics, pharmacy, philosophies, photonics, physics, plants, plasma, polymers, polysaccharides, preprints , proceedings, processes, proteomes, psych, publications, quantumrep, quaternary, qubs, reactions, recycling, religions, remotesensing, reports, resources, risks, robotics, safety, sci, scipharm, sensors, separations, sexes, signals, sinusitis, smartcities, sna, societies, socsci, soilsystems, sports, standards, stats, surfaces, surgeries, sustainability, symmetry, systems, technologies, test, toxics, toxins, tropicalmed, universe, urbansci, vaccines, vehicles, vetsci, vibration, viruses, vision, water, wem, wevj

%---------
% article
%---------
% The default type of manuscript is "article", but can be replaced by:
% abstract, addendum, article, benchmark, book, bookreview, briefreport, casereport, changes, comment, commentary, communication, conceptpaper, conferenceproceedings, correction, conferencereport, expressionofconcern, extendedabstract, meetingreport, creative, datadescriptor, discussion, editorial, essay, erratum, hypothesis, interestingimages, letter, meetingreport, newbookreceived, obituary, opinion, projectreport, reply, retraction, review, perspective, protocol, shortnote, supfile, technicalnote, viewpoint
% supfile = supplementary materials

%----------
% submit
%----------
% The class option "submit" will be changed to "accept" by the Editorial Office when the paper is accepted. This will only make changes to the frontpage (e.g., the logo of the journal will get visible), the headings, and the copyright information. Also, line numbering will be removed. Journal info and pagination for accepted papers will also be assigned by the Editorial Office.

%------------------
% moreauthors
%------------------
% If there is only one author the class option oneauthor should be used. Otherwise use the class option moreauthors.

%---------
% pdftex
%---------
% The option pdftex is for use with pdfLaTeX. If eps figures are used, remove the option pdftex and use LaTeX and dvi2pdf.

%=================================================================
\firstpage{1}
\makeatletter
\setcounter{page}{\@firstpage}
\makeatother
\pubvolume{xx}
\issuenum{1}
\articlenumber{5}
\pubyear{2019}
\copyrightyear{2019}
%\externaleditor{Academic Editor: name}
\history{Received: date; Accepted: date; Published: date}
\updates{yes} % If there is an update available, un-comment this line

%% MDPI internal command: uncomment if new journal that already uses continuous page numbers
%\continuouspages{yes}

%------------------------------------------------------------------
% The following line should be uncommented if the LaTeX file is uploaded to arXiv.org
%\pdfoutput=1

%=================================================================
% Add packages and commands here. The following packages are loaded in our class file: fontenc, calc, indentfirst, fancyhdr, graphicx, lastpage, ifthen, lineno, float, amsmath, setspace, enumitem, mathpazo, booktabs, titlesec, etoolbox, amsthm, hyphenat, natbib, hyperref, footmisc, geometry, caption, url, mdframed, tabto, soul, multirow, microtype, tikz

%=================================================================
%% Please use the following mathematics environments: Theorem, Lemma, Corollary, Proposition, Characterization, Property, Problem, Example, ExamplesandDefinitions, Hypothesis, Remark, Definition
%% For proofs, please use the proof environment (the amsthm package is loaded by the MDPI class).

%=================================================================
% Full title of the paper (Capitalized)
\Title{Gráfico de control T2 Hotelling para variables cualitativas}

% Authors, for the paper (add full first names)
\Author{Wilson
Rojas-Preciado$^{1,2}$\href{https://orcid.org/0000-0003-1614-698X}{\orcidicon}, Mauricio
Rojas-Campuzano$^{3}$\href{https://orcid.org/0000-0001-8000-9432}{\orcidicon}, Purificación
Galindo-Villardón$^{2}$\href{https://orcid.org/0000-0001-6977-7545}{\orcidicon}, Omar
Ruiz-Barzola$^{3}$\href{https://orcid.org/0000-0001-8206-1744}{\orcidicon}, $^{}$, $^{}$, $^{}$}

% Authors, for metadata in PDF
\AuthorNames{Wilson Rojas-Preciado, Mauricio
Rojas-Campuzano, Purificación Galindo-Villardón, Omar
Ruiz-Barzola, , , }

% Affiliations / Addresses (Add [1] after \address if there is only one affiliation.)
\address{%
}
% Contact information of the corresponding author
\corres{Correspondence: \href{mailto:wrojas@utmachala.edu.ec}{\nolinkurl{wrojas@utmachala.edu.ec}};
Tel.: +593-992-83-3719}

% Current address and/or shared authorship
\firstnote{Current address: Updated affiliation}
\secondnote{These authors contributed equally to this work.}






% The commands \thirdnote{} till \eighthnote{} are available for further notes

% Simple summary
\simplesumm{A Simple summary goes here.}

% Abstract (Do not insert blank lines, i.e. \\)
\abstract{Abstract}

% Keywords
\keyword{keyword 1; keyword 2; keyword 3 (list three to ten pertinent
keywords specific to the article, yet reasonably common within the
subject discipline.).}

% The fields PACS, MSC, and JEL may be left empty or commented out if not applicable
%\PACS{J0101}
%\MSC{}
%\JEL{}

%%%%%%%%%%%%%%%%%%%%%%%%%%%%%%%%%%%%%%%%%%
% Only for the journal Diversity
%\LSID{\url{http://}}

%%%%%%%%%%%%%%%%%%%%%%%%%%%%%%%%%%%%%%%%%%
% Only for the journal Applied Sciences:
%\featuredapplication{Authors are encouraged to provide a concise description of the specific application or a potential application of the work. This section is not mandatory.}
%%%%%%%%%%%%%%%%%%%%%%%%%%%%%%%%%%%%%%%%%%

%%%%%%%%%%%%%%%%%%%%%%%%%%%%%%%%%%%%%%%%%%
% Only for the journal Data:
%\dataset{DOI number or link to the deposited data set in cases where the data set is published or set to be published separately. If the data set is submitted and will be published as a supplement to this paper in the journal Data, this field will be filled by the editors of the journal. In this case, please make sure to submit the data set as a supplement when entering your manuscript into our manuscript editorial system.}

%\datasetlicense{license under which the data set is made available (CC0, CC-BY, CC-BY-SA, CC-BY-NC, etc.)}

%%%%%%%%%%%%%%%%%%%%%%%%%%%%%%%%%%%%%%%%%%
% Only for the journal Toxins
%\keycontribution{The breakthroughs or highlights of the manuscript. Authors can write one or two sentences to describe the most important part of the paper.}

%\setcounter{secnumdepth}{4}
%%%%%%%%%%%%%%%%%%%%%%%%%%%%%%%%%%%%%%%%%%

% Pandoc citation processing

\usepackage{subfig}

\begin{document}
%%%%%%%%%%%%%%%%%%%%%%%%%%%%%%%%%%%%%%%%%%

\hypertarget{introduction}{%
\section{Introduction}\label{introduction}}

Los gráficos de control constituyen una de las herramientas más
importantes para definir límites y parámetros óptimos de los procesos,
así como para controlar la calidad de los productos mediante la
reducción de la variabilidad. El uso de gráficos de control facilita la
evaluación del comportamiento de las variables del proceso y contribuye
al logro de los objetivos planificados.

La variación de los procesos se entiende como la diversidad de
resultados que genera un grupo de variables de un proceso, su monitoreo
es un objetivo clave del control estadístico, por lo tanto, es necesario
entender los tipos y motivos de la variabilidad. Para ello es preciso
registrar de manera sistemática y adecuada diferentes variables del
proceso que se desea controlar, como las propiedades de los insumos, las
condiciones de operación de los equipos, las competencias del personal
que maneja los procesos, además de las características de los productos,
la satisfacción de los usuarios, el cumplimiento de requisitos, entre
otras.

El pionero del control estadístico de procesos fue Walter Shewhart.
Estableció las diferencias entre la variabilidad natural o común,
presente en todos los procesos, y la provocada por causas asignables o
especiales, que pueden llevarlos a un estado de fuera de control. Señaló
que un proceso está en control estadístico cuando trabaja sólo con
causas comunes de variación. Propuso los primeros gráficos de control
para variables de tipo continuo y para variables de atributos
\citep{Gutierrez2013}.

El control estadístico de procesos mediante gráficos de control permitió
a las organizaciones monitorear el comportamiento de una variable a la
vez, no obstante, las organizaciones requirieron, con el pasar del
tiempo, el análisis de varias características de calidad de forma
simultánea, abriendo la puerta al control estadístico de procesos desde
una perspectiva multivariante \citep{ramos2017}. Para facilitar el
control de calidad de procesos es común el uso de gráficas de control
que recolectan abundante información en diversas variables de forma
simultánea, su análisis permite caracterizar los diferentes tipos de
variables que afectan la calidad y explican su comportamiento a lo largo
del tiempo \citep{li2012}.

Hay una variedad de gráficos de control de procesos desde la perspectiva
multivariante, entre los clásicos están el Gráfico \(T^2\) de Hotelling
\citep{hotelling1947}, el Multivariate Exponentially Weighted Moving --
MEWMA \citep{lowry1992}, el Multivariate Cumulative Sum Control Chart --
MCUSUM \citep{Crosier1988}. Con el transcurso del tiempo se hicieron
diversos aportes para mejorar el rendimiento de estos gráficos, entre
los más destacados están Gráfico de control \(T^2\) con tamaños de
muestra adaptables \citep{Aparisi1996}, Gráfico de control \(T^2\) con
intervalos de muestreo variables \citep{Aparisi2001}, Gráfico de control
\(T^2\) con líneas de advertencia dobles \citep{Faraz2006}, Gráfico de
control robusto \citep{shabbak2012}, Gráficos de control basados en
modelos de minería de datos para procesos multivariantes y
autocorrelacionados \citep{kim2012}, Gráficos de control de calidad
multivariantes con dimensión variable \citep{ruiz2013}, Gráfico de
control para el coeficiente de variación multivariante
\citep{yeong2016}.

Además de estos gráficos de control para entornos paramétricos, se
desarrollaron otros para datos numéricos y cualitativos en entornos
multivariantes no paramétricos, entre ellos el Gráfico de control
multivariante basado en la distancia de Gower para una combinación de
variables continuas y cualitativas \citep{Tuerhong2014}, Gráfico de
control multivariante basado en la combinación de PCA para
características de calidad de atributos y variables
\citep{Muhammad2018}, Gráfico de control multivariante no paramétrico
basado en la ponderación de novedad sensible a la densidad para procesos
no normales \citep{liu2020}, Gráfico de control de deméritos con
clustering difuso de c-medias \citep{yilmaz2020}, Gráfico de control
basado en ACP que utiliza máquinas de vectores de soporte para
distribuciones no normales multivariadas \citep{Farokhnia}, Gráfico
CUSUM no paramétrico para monitorear procesos multivariados
correlacionados en serie \citep{xue2020}, Gráfico de control
multivariante basado en Kernel PCA para monitorear características de
calidad de atributos y variables mixtas \citep{Ahsan2020}, Gráfico
\(T^2\) basado en la combinación de PCA para datos continuos y
cualitativos con detección de datos atípicos \citep{Ahsan2021}.

Como se puede observar, la literatura científica es abundante en lo
referente a gráficos de control en entornos multivariantes paramétricos
y no paramétricos para datos numéricos y, en los últimos años, para
datos mixtos (numéricos y cualitativos), sin embargo, no se puede decir
lo mismo de las publicaciones sobre gráficos de control multivariantes
para datos cualitativos.

En el estudio de los procesos que se desarrollan en el entorno
social-educativo y que explican el comportamiento de variables como el
rendimiento académico, tasas de graduación o deserción, producción
científica, porcentajes de matrícula de nuevo ingreso, entre otros, se
maneja con mucha frecuencia variables cualitativas. No es que estén
ausentes los datos cuantitativos, sino que, en las bases de datos que se
utilizan para estos análisis, abundan las variables cualitativas
nominales y ordinales sobre las de tipo numérico, algunos ejemplos de
datos de los estudiantes son: sexo, lugar de procedencia,
autodenominación étnica, grado académico de los padres, tipo de
institución educativa de procedencia (fiscal, particular, municipal);
ejemplos de datos de las instituciones son: tipo de sostenimiento
económico, jornada, modalidad, campo de estudio, niveles (tecnológico,
grado y postgrado), tipo de infraestructura; ejemplos asociados a datos
de los profesores son: titularidad, dedicación, grado académico, grado
en el escalafón, discapacidad, entre otros.

\citet{perez2004} señala que al observar muchas variables sobre una
muestra es presumible que una parte de la información recogida pueda ser
redundante o que sea excesiva, en cuyo caso los métodos multivariantes
de reducción de la dimensión tratan de eliminarla combinando muchas
variables observadas para quedarse con pocas variables ficticias que,
aunque no observadas, sean combinación de las reales y sinteticen la
mayor parte de la información contenida en sus datos. En este caso se
deberá tener en cuenta el tipo de variables que maneja. Si son variables
cuantitativas las técnicas que le permiten este tratamiento pueden ser
el Análisis de componentes principales
\citep{Person1901, Hotelling1933}, el Análisis factorial
\citep{ch1904, thurstone1947, kaiser1958}, mientras que, si se trata de
variables cualitativas, es recomendable la aplicación de un Análisis de
correspondencias múltiple, Análisis de homogeneidad o un Análisis de
Escalamiento multidimensional.

\hypertarget{anuxe1lisis-de-correspondencias}{%
\subsection{Análisis de
Correspondencias}\label{anuxe1lisis-de-correspondencias}}

El tratamiento multivariante de variables cualitativas requiere un
proceso metodológico distinto, uno de los más representativos es el
Análisis de Correspondencias \citep{Benzecri}. Según \citep{perez2004},
este análisis implica estudios de similaridad o disimilaridad entre
categorías, se debe cuantificar la diferencia o distancia entre ellas
sumando las diferencias cuadráticas relativas entre las frecuencias de
las distribuciones de las variables analizadas, lo que conduce al
concepto de la \(\chi^2\). Así, el análisis de correspondencias puede
considerase como un análisis de componentes principales aplicado a
variables cualitativas que, al no poder utilizar correlaciones, se basa
en la distancia no euclídea de la\(\chi^2\). En el enfoque francés del
Análisis de Correspondencias, que se caracteriza por el énfasis en la
geometría, el análisis de una tabla cruzada se llama análisis de
correspondencia (AC) y el análisis de una colección de matrices
indicadoras, se denomina análisis de correspondencia múltiple (ACM)
\citep{michailidis1998}. En contextos anglosajones, el ACM es conocido
como Análisis de Homogeneidad o Escalamiento Dual, especialmente en
psicometría.

\hypertarget{anuxe1lisis-de-homogeneidad}{%
\subsection{Análisis de
Homogeneidad}\label{anuxe1lisis-de-homogeneidad}}

El Análisis de Homogeneidad, Homogeneous Alternating Least Squares
(HOMALS), es un modelo de la familia de modelos matemáticos del
Escalamiento óptimo del sistema Gifi \citep{Gifi1990}, el cual comprende
una serie de técnicas exploratorias de análisis multivariado no lineal.
Igual que el ACM, HOMALS se considera una forma de Análisis de
Componentes Principales para datos cualitativos. El Análisis de
Homogeneidad representa los objetos analizados mediante puntos en el
modelo espacial, sus características más relevantes se presentan en las
relaciones geométricas entre los puntos, para ello, es necesario la
cuantificación de datos cualitativos \citep{Lopez2014}. El uso de
variables cualitativas no es particularmente restrictivo, ya que una
variable numérica continua se puede considerar como una variable
cualitativa con un gran número de categorías. HOMALS se diferencia del
el ACM en que éste utiliza la función de Descomposición de valores
propios mientras que el Análisis de Homogeneidad utiliza Mínimos
Cuadrados Alternos, lo que se conoce en la literatura como la Solución
de Homals \citep{michailidis1998}.

\hypertarget{escalamiento-multidimensional}{%
\subsection{Escalamiento
multidimensional}\label{escalamiento-multidimensional}}

Otra de las técnicas multivariantes para el tratamiento de variables
cualitativas es el escalamiento multidimensional (EMD)
\citep{torgerson1952, shepard1962}. Se define como una técnica que
elabora una representación gráfica que permite conocer la imagen que los
individuos se crean de un conjunto de objetos por posicionamiento de
cada uno en relación con los demás, (mapa perceptual). El EMD trata de
encontrar la estructura de un conjunto de medidas de distancia entre
objetos o casos. Esto se logra asignando las observaciones a posiciones
específicas en un espacio conceptual, normalmente de dos o tres
dimensiones, de modo que las distancias entre los puntos en el espacio
concuerden al máximo con las disimilaridades dadas. Las dimensiones de
este espacio conceptual se pueden interpretar para favorecer la
comprensión de los datos, inclusive si son valoraciones subjetivas de
disimilaridad entre objetos o conceptos. Dado que el EMD permite
calcular las distancias a partir de los datos multivariados, si las
variables se han medido objetivamente, se lo puede utilizar como técnica
de reducción de datos \citep{perez2004}.

Las variables que utiliza esta técnica pueden ser métricas o no
métricas. El escalamiento multidimensional las transforma en distancias
entre los objetos en un espacio de dimensiones múltiples, de modo que
objetos que aparecen situados más próximos entre sí son percibidos como
más similares por los sujetos.

\hypertarget{materials-and-methods}{%
\section{Materials and Methods}\label{materials-and-methods}}

\hypertarget{notaciuxf3n}{%
\subsection{Notación}\label{notaciuxf3n}}

\begin{table}[!ht]
\begin{center}
 \begin{tabular}{||c ||c |c ||} 
 \hline
 Caracteres & Tipo & Ejemplo \\
 \hline\hline
 Escalares & Letras en minúscula. & $v,\lambda$\\
\hline
Vectores & Letras en minúscula y en negrita. & $\mathbf{v},\mathbf{u}$\\
\hline
Matrices & Letras en mayúscula y en negrita. & $\mathbf{V},\mathbf{X}$\\
\hline
Matrices de tres vías (Cubos de datos) & Letras con doble trazo en mayúscula. & $\mathbb{C},\mathbb{X}$\\
\hline
\end{tabular}\caption{Notación}
\label{tab:notacion}
\end{center}
\end{table}

A lo largo del artículo se utilizarán letras para hacer referencia a
parámetros necesarios, se los enuncia a continuación:

\begin{table}[!ht]
\begin{center}
 \begin{tabular}{||c ||c | c ||} 
 \hline
 Letra &  Significado & Especificación\\
 \hline\hline
 p & Número de dimensiones &\\
\hline
 K & Número total de tablas (Especifica la profundidad del cubo de datos) & \\
 \hline
 $k$ & Índice de tabla &  k=1,2,...,K\\
  \hline
 $T$ & Índice de matriz transpuesta &  $\mathbf{X^{T}}$\\
\hline
 n & Tamaño muestral de las K tablas &\\
\hline
\end{tabular}\caption{Notación}
\label{tab:notacion}
\end{center}
\end{table}

\hypertarget{anuxe1lisis-de-correspondencia-muxfaltiple-mca}{%
\subsection{Análisis de Correspondencia Múltiple
(MCA)}\label{anuxe1lisis-de-correspondencia-muxfaltiple-mca}}

El análisis de correspondencias múltiples (MCA) es una generalización
del análisis de correspondencias simple o binario, donde se incluyen más
variables cualitativas Se obtiene al realizar el análisis de
correspondencias simple a una tabla disyuntiva completa, conocida como
la tabla de Burt.

Se tiene una matriz de datos con \(p\) variables cualitativas, cada una
con h categorías (h \textgreater1). En el ejemplo que se desarrolla para
esta investigación, se dispone de una base de datos (nombre de la base
de datos) constituida por 10 tablas, cada una tiene 10 variables y cada
variable, 3 categorías (Alto, Medio y Bajo).

\begin{table}[!ht]
\begin{center}
 \begin{tabular}{||c c c c||} 
 \hline
 $V_{1}$ & $V_{2}$ & $\cdots$ & $V_{p}$ \\ [0.5ex] 
 \hline\hline
 Alto & Medio & $\cdots$ & Medio\\
 \hline
Medio & Bajo & $\cdots$ & Alto\\
\hline
\vdots & $\vdots$ & $\vdots$ & $\vdots$\\
\hline
Bajo & Alto & $\cdots$ & Bajo \\ [1ex] 
 \hline
\end{tabular}\caption{Matriz inicial}
\label{tab:inicial}
\end{center}
\end{table}

Esta matriz es equivalente a la matriz disyuntiva \(Z\), que desglosa
las variables en cada una de sus modalidades y se registra la ocurrencia
de eventos de forma binaria.

\begin{table}[!ht]
\begin{center}
 \begin{tabular}{||p{1cm}p{1cm}p{1cm}||p{1cm}p{1cm} p{1cm} ||p{1cm} ||p{1cm} p{1cm} p{1cm} ||} 
 \hline
 $V_{1}:Alto$ &$V_{1}:Medio$ &$V_{1}:Bajo$ & $V_{2}:Alto$ & $V_{2}:Medio$ & $V_{2}:Bajo$ & $\cdots$ & $V_{p}:Alto$ & $V_{p}:Medio$ & $V_{p}:Bajo$ \\ [0.5ex] 
 \hline\hline
 1 & 0 & 0 & 1 & 0 & 0 & $\cdots$ & 0 & 1 & 0 \\ [0.2ex] 
 \hline
 0 & 1 & 0 & 0 & 1 & 0 & $\cdots$ & 1 & 0 & 0 \\ 
\hline
 0 & 0 & 1 & 0 & 0 &  1 & $\cdots$ & 0 & 0 & 1 \\ 
\hline
 $\vdots$ & $\vdots$ & $\vdots$ & $\vdots$ & $\vdots$ &  $\vdots$ & $\ddots$ & $\vdots$ & $\vdots$ & $\vdots$ \\ 
\hline
 1 & 0 & 0 & 1 & 0 & 0 & $\cdots$ &1 & 0 & 0 \\  
 \hline
\end{tabular}
\caption{Matriz disyuntiva Z}
\label{tab:z}
\end{center}
\end{table}

La tabla de Burt viene dada por:

\begin{equation}
\mathbf{B}=\mathbf{Z'}\mathbf{Z}
\label{eq:Burt}
\end{equation}

\emph{La matriz de Burt se construye por superposición de cajas. En los
bloques diagonales aparecen matrices diagonales conteniendo las
frecuencias marginales de cada una de las variables analizadas. Fuera de
la diagonal aparecen las tablas de frecuencia cruzadas correspondientes
a todas las combinaciones 2 a 2 de las variables analizadas}\\
Para realizar el análisis de correspondencias múltiples se parte de la
matriz de Burt, obtenida con la ecuación \ref{eq:Burt}. Esta matriz está
formada por las frecuencias absolutas, éstas se transforman en
frecuencias relativas, dividiendo los valores de la matriz por la
frecuencia total, dando lugar a una nueva matriz que se denominará
\textbf{P}.

\begin{table}[!ht]
\begin{center}
 \begin{tabular}{| p{1.7cm} ||p{1cm}p{1cm}p{1cm}||p{1cm}p{1cm} p{1cm} ||p{1cm} ||p{1.3cm} p{1cm} p{1cm} ||} 
 \hline
  & $V_{1}:Alto$ &$V_{1}:Medio$ &$V_{1}:Bajo$ & $V_{2}:Alto$ & $V_{2}:Medio$ & $V_{2}:Bajo$ & $\cdots$ & $V_{p}:Alto$ & $V_{p}:Medio$ & $V_{p}:Bajo$ \\ [0.5ex] 
 \hline\hline
 $V_{1}:Alto$ & $b_{1,1}$ & 0 & 0  & $b_{1,4}$ & $b_{1,5}$ & $b_{1,6}$ & $\cdots$ & $b_{1,3p-2}$ & $b_{1,3p-1}$ & $b_{1,3p}$ \\
 $V_{1}:Medio$ & 0 & $b_{2,2}$ & 0 & $b_{2,4}$ & $b_{2,5}$ & $b_{2,6}$ & $\cdots$ & $b_{2,3p-2}$ & $b_{2,3p-1}$ & $b_{2,3p}$ \\ 
 $V_{1}:Bajo$ & 0 & 0 & $b_{3,3}$  & $b_{3,4}$ & $b_{3,5}$ & $b_{3,6}$ & $\cdots$ & $b_{3,3p-2}$ & $b_{3,3p-1}$ & $b_{3,3p}$ \\ 
\hline\hline
  $V_{2}:Alto$ & $b_{4,1}$ & $b_{4,2}$ & $b_{4,3}$ & $b_{4,4}$ & 0 & 0 & $\cdots$ & $b_{4,3p-2}$ & $b_{4,3p-1}$ & $b_{4,3p}$ \\
 $V_{2}:Medio$ & $b_{5,1}$ & $b_{5,2}$ & $b_{5,3}$ & 0 & $b_{5,5}$ & 0 & $\cdots$ & $b_{5,3p-2}$ & $b_{5,3p-1}$ & $b_{5,3p}$ \\ 
 $V_{2}:Bajo$ &  $b_{6,1}$ & $b_{6,2}$ & $b_{6,3}$ & 0 & 0 &  $b_{6,6}$ & $\cdots$& $b_{6,3p-2}$ & $b_{6,3p-1}$ & $b_{6,3p}$ \\ 
\hline\hline
 
 $\vdots$ & $\vdots$ & $\vdots$ & $\vdots$ & $\vdots$ & $\vdots$ & $\vdots$ & $\ddots$ & $\vdots$ & $\vdots$ & $\vdots$ \\ 
 
\hline\hline
 $V_{p}:Alto$  & $b_{3p-2,1}$   & $b_{3p-2,2}$   & $b_{3p-2,3}$   & $b_{3p-2,4}$   & $b_{3p-2,5}$   & $b_{3p-2,6}$    & $\cdots$ & $b_{3p-2,3p-2}$ & 0 & 0 \\
 $V_{p}:Medio$ & $b_{3p-1,1}$ & $b_{3p-1,2}$ & $b_{3p-1,3}$ & $b_{3p-1,4}$ & $b_{3p-1,5}$ & $b_{3p-1,6}$  & $\cdots$ & 0 & $b_{3p-1,3p-1}$ & 0 \\ 
 $V_{p}:Bajo$  & $b_{3p,1}$ & $b_{3p,2}$ & $b_{3p,3}$ & $b_{3p,4}$ & $b_{3p,5}$ & $b_{3p,6}$  & $\cdots$ & 0 & 0 & $b_{3p,3p}$ \\ 
\hline
\end{tabular}
\caption{P: Tabla de contingencia de Burt en frecuencias relativas}
\label{tab:p}
\end{center}
\end{table}

Se obtienen las marginales de las filas \emph{(mf)} y de las columnas
\emph{(mc)} de la matriz \textbf{P} (Tabla \ref{tab:p}). A estos
vectores se los conoce también como \emph{Masas de fila y columna},
respectivamente.

\begin{table}[!ht]
\begin{center}
 \begin{tabular}{||p{1cm}p{1cm}p{1cm}||p{1cm}p{1cm} p{1cm} ||p{1cm} ||p{1cm} p{1cm} p{1cm} ||} 
 \hline
 $V_{1}:Alto$ &$V_{1}:Medio$ &$V_{1}:Bajo$ & $V_{2}:Alto$ & $V_{2}:Medio$ & $V_{2}:Bajo$ & $\cdots$ & $V_{p}:Alto$ & $V_{p}:Medio$ & $V_{p}:Bajo$ \\ [0.5ex] 
 \hline
    $b_{\bullet,1}$ & $b_{\bullet,2}$ & $b_{\bullet,3}$ & $b_{\bullet,4}$ & $b_{\bullet,5}$ & $b_{\bullet,6}$ & $\cdots$ & $b_{\bullet,3p-2}$ & $b_{\bullet,3p-1}$ & $b_{\bullet,3p}$ \\ [0.5ex] 
 \hline
\end{tabular}
\caption{Frecuencias marginales de las filas. (mf)}
\label{tab:margfilas}
\end{center}
\end{table}

\begin{table}[h!]
\begin{center}
 \begin{tabular}{||p{1cm}p{1cm}p{1cm}||p{1cm}p{1cm} p{1cm} ||p{1cm} ||p{1cm} p{1cm} p{1cm} ||} 
 \hline
 $V_{1}:Alto$ &$V_{1}:Medio$ &$V_{1}:Bajo$ & $V_{2}:Alto$ & $V_{2}:Medio$ & $V_{2}:Bajo$ & $\cdots$ & $V_{p}:Alto$ & $V_{p}:Medio$ & $V_{p}:Bajo$ \\ [0.5ex] 
 \hline
    $b_{\bullet,1}$ & $b_{\bullet,2}$ & $b_{\bullet,3}$ & $b_{\bullet,4}$ & $b_{\bullet,5}$ & $b_{\bullet,6}$ & $\cdots$ & $b_{\bullet,3p-2}$ & $b_{\bullet,3p-1}$ & $b_{\bullet,3p}$ \\ [0.5ex] 
 \hline
\end{tabular}
\caption{Frecuencias marginales de las columnas. (mc)}
\label{tab:margcolumnas}
\end{center}
\end{table}

Se obtiene la matriz de residuos estandarizados \textbf{S}.

\begin{equation}
\mathbf{S}=\mathbf{D_{fila}}^{-\frac{1}{2}}(\mathbf{P}-\mathbf{mf} \hspace{0.2cm} \mathbf{mc'})\mathbf{D_{columna}}^{-\frac{1}{2}}
\label{eq:s}
\end{equation} donde:

\begin{itemize}
\tightlist
\item
  \(\mathbf{D_{fila}}\) es una matriz diagonal que contiene las masas de
  las filas.\\
\item
  \(\mathbf{D_{columna}}\) es una matriz diagonal que contiene las masas
  de las columnas
\end{itemize}

Se aplica descomposición singular (SVD) a la matriz \textbf{S} (Ecuación
\ref{eq:s}):

\begin{equation}
\mathbf{S}=\mathbf{U}\mathbf{D}\mathbf{V'}
\label{eq:svd}
\end{equation} donde:

\begin{itemize}
\tightlist
\item
  \(\mathbf{U}\) y \(\mathbf{V}\) son matrices ortogonales.\\
\item
  \(\mathbf{D}\) es una matriz diagonal que contiene los valores
  singulares.
\end{itemize}

Para encontrar las coordenadas estandarizadas se aplica lo siguiente:

\begin{equation}
\mathbf{X}=\mathbf{D_{fila}}^{-\frac{1}{2}} \mathbf{U}
\label{eq:xcoor}
\end{equation}

\begin{equation}
\mathbf{Y}=\mathbf{D_{columna}}^{-\frac{1}{2}} \mathbf{V}
\label{eq:ycoor}
\end{equation}

Para los fines necesarios, utilizaremos las coordenadas de las columnas
(Tabla \ref{tab:colcoor}).

\begin{table}[!ht]
\begin{center}
 \begin{tabular}{|| c ||c c c c||} 
 \hline
 & $Dim_{1}$      & $Dim_{2}$ & $\cdots$ & $Dim_{3p}$ \\ [0.5ex] 
 \hline\hline
  $V_{1}:Alto$    & ${v_{1}d_{1}}_{alto}$& ${v_{1}d_{1}}_{alto}$  & $\cdots$ & ${v_{1}d_{p}}_{alto}$\\
 \hline
 $V_{1}:Medio$    &${v_{1}d_{1}}_{medio}$ & ${v_{1}d_{1}}_{medio}$ & $\cdots$ & ${v_{1}d_{p}}_{medio}$\\
\hline
 $V_{1}:Bajo$     &${v_{1}d_{1}}_{bajo}$ & ${v_{1}d_{1}}_{bajo}$  &$\cdots$ & ${v_{1}d_{p}}_{bajo}$\\
\hline
\vdots & $\vdots$ & $\vdots$  &$\ddots$& $\vdots$\\
\hline
 $V_{p}:Bajo$     &${v_{p}d_{1}}_{bajo}$ & ${v_{p}d_{1}}_{bajo}$ & $\cdots$ & ${v_{p} d_{p}} _{bajo}$ \\ [1ex] 
 \hline
\end{tabular}\caption{Coordenadas estandarizadas de las columnas.}
\label{tab:colcoor}
\end{center}
\end{table}

\hypertarget{generalizaciuxf3n-a-k-tablas}{%
\subsection{Generalización a K
tablas}\label{generalizaciuxf3n-a-k-tablas}}

Si se tienen K tablas, con la misma estructura de la tabla
\ref{tab:inicial}, como se visualiza en la figura \ref{fig:ktables},
abordamos el enfoque del análisis factorial múltiple (MFA). \citet{AFM}
indica que el MFA utiliza análisis de correspondencia múltiple cuando se
trata de variables cualitativas. El procedimiento implica la realización
de un MCA por cada tabla y dividirlo por su primer valor propio con la
finalidad de obtener K grupos normalizados. Posteriormente se consideran
todas las tablas y se realiza un MCA global.

\begin{figure}[!ht]



\begin{center}\includegraphics[width=0.4\linewidth,]{ktables} \end{center}

\caption{K tablas con el formato inicial.}

\label{fig:ktables}
\end{figure}

La generalización a K tablas del procedimiento del MCA, se presenta en
la Figura \ref{fig:MCAk}

\begin{figure}[!h]


\begin{center}\includegraphics[width=0.9\linewidth,]{ktablesMCA} \end{center}

\caption{Procedimiento del MCA para K tablas}

\label{fig:MCAk}
\end{figure}

Llamaremos \(C\) a cada tabla de coordenadas. Con la finalidad de
detectar la magnitud de las variables latentes, su aporte neto a las
variables, se trata la matriz \(C\) con valor absoluto.

\hypertarget{aporte-del-anuxe1lisis-factorial-muxfaltiple-mfa}{%
\subsection{Aporte del Análisis Factorial Múltiple
(MFA)}\label{aporte-del-anuxe1lisis-factorial-muxfaltiple-mfa}}

Una vez que se tienen las coordenadas de las columnas, se procede a
realizar la normalización, característica del procedimiento MFA.

Sea \(\lambda_{1}^{k}\) el primer valor propio obtenido de la
descomposición singular de la k-ésima tabla C. Normalizaremos la tabla
multiplicándola por \(1/\lambda_{1}^{k}\). Con esto se obtiene la tabla
\(C^{'}\), que corresponde a la tabla de coordenadas normalizadas.\\
Individualmente, para el caso de la matriz k, se tendría la siguiente
expresión.

\begin{equation}
\mathbf{C'_k}=\frac{1}{\lambda_{k}^1} \mathbf{C_k}
\label{eq:Cprimak}
\end{equation}

Aglomerando las matrices normalizadas \(C^{'}\) en una sola, se tiene la
matriz \(\mathbb{C}^{'}\). Esta contiene todos los elementos de las k
tablas.

\begin{equation}
\mathbf{\mathbb{C^{'}}}=[\mathbf{C_1^{'}}|\mathbf{C_2^{'}}|,...,|\mathbf{C'_{K}}]^{T}
\label{eq:Cprima}
\end{equation}

La normalización que realiza el MFA se encarga de ponderar las k tablas,
con el objetivo de evitar alguna descompensación al momento de realizar
el análisis conjunto de las tablas.

\hypertarget{gruxe1fico-de-control}{%
\subsection{Gráfico de control}\label{gruxe1fico-de-control}}

Para definir el gráfico de control \(T^2\) Hotelling se deben tomar las
siguientes consideraciones:

\begin{itemize}
\tightlist
\item
  La tabla \(\mathbb{C}^{'}\) (Ecuación \ref{eq:Cprima}) se denomina
  Consenso, sirve como referente para el escenario \emph{bajo control},
  y de la cual se obtiene \(\mu_{0}\) y \(\mathbf{S_0}\).\\
\item
  Cada matriz \(\mathbf{C'_k}\) tiene el mismo número de filas (n) y
  columnas (p) (individuos y variables).
\item
  El vector de medias \(\mathbf{\mu_k}\) está atado a la tabla
  \(\mathbf{C'_k}\), es decir, el gráfico de control estará en función
  de las diferencias entre las matrices \(\mathbf{C'_k}\) y la matriz
  consenso \(\mathbf{\mathbb{C^{'}}}\).
\item
  Las matrices \(\mathbf{C'_k}\) siguen una distribución normal
  multivariante con vector de medias \(\mu_{k}\) y matriz de covarianzas
  \(\mathbf{S_k}\).
\end{itemize}

Con esto, se obtiene el estadístico \(T^2\):

\begin{equation}
T^2=n (\mu_{k}-\mu_{0})'\mathbf{\Sigma_{0}^{-1}}(\mu_{k}-\mu_{0})
\label{eq:T2}
\end{equation}

Se sabe que, bajo control, el \(T^2\) se distribuye como una
Chi-cuadrado con p grados de libertad \(\chi^2_p\). En este caso se
puede aplicar este principio, ya que utilizamos la matriz consenso
(\(\mathbb{C}^{'}\)), que representa al escenario bajo control.

Dado que este gráfico de control está basado en distancias de
Mahalanobis ponderadas, sólo tiene límite de control superior. Este
viene dado por la ecuación \ref{eq:UCL}

\begin{equation}
UCL=\chi^2_{\alpha,p}
\label{eq:UCL}
\end{equation}

donde, \(\alpha\) es la significancia predeterminada considerando el
número de variables \(p\).\\
\(p\) es el número de dimensiones.

\hypertarget{tabla-posterior}{%
\subsection{Tabla posterior}\label{tabla-posterior}}

Con la finalidad de detectar las potenciales categorías responsables de
que un punto en el gráfico \(T^2\) de Hotelling para variables
cualitativas se encuentre fuera de control, se propone una tabla que
presenta las anomalías de cada categoría en cada variable, comparando
las masas de columna de la tabla \(k\) y las masas de columna de la
tabla consenso por medio de distancias \(\chi^2\) que proporcionan un
valor p, aportando a la interpretación.

\hypertarget{complemento-computacional}{%
\section{Complemento computacional}\label{complemento-computacional}}

Para facilitar la distribución y aplicación del método propuesto, se ha
desarrollado un paquete reproducible en R. El paquete \textbf{T2Qv}
utiliza la metodología expuesta en este artículo y la lleva a un entorno
práctico, permite visualizar los resultados de forma plana o
interactiva, además, presenta un panel Shiny que contiene todas las
funciones individuales en un mismo espacio.

\hypertarget{disponibilidad}{%
\subsection{Disponibilidad}\label{disponibilidad}}

El paquete está disponible en GitHub, la descarga se la puede realizar
de la siguiente forma:

\begin{verbatim}
install.packages("devtools")
devtools::install_github("JavierRojasC/T2Qv")
\end{verbatim}

\hypertarget{el-paquete-t2qv}{%
\subsection{El paquete: T2Qv}\label{el-paquete-t2qv}}

\begin{figure}[!ht]



\begin{center}\includegraphics[width=0.6\linewidth,]{DescrPack} \end{center}

\caption{Documentación del paquete T2Qv}

\label{fig:documentation}
\end{figure}

Las funciones que contiene el paquete y su descripción se enuncian en la
tabla \ref{tab:functions}.

\begin{table}[!ht]
\begin{center}
 \begin{tabular}{||c  m{35em}||} 
 \hline
  Función & Descripción \\ [0.5ex] 
 \hline\hline
 T2 qualitative & Multivariate control chart T2 Hotelling applicable for qualitative variables.\\
 \hline
  ACMconsensus & Multiple correspondence analysis applied to a consensus table.\\
\hline
  ACMpoint & Multiple correspondence analysis applied to a specific table.\\
\hline
  ChiSq variable & Contains Chi square distance between the column masses of the table specified in PointTable and the consensus table. It allows to identify which mode is responsible for the anomaly in the table in which it is located. \\ [1ex] 
  \hline
  Full Panel & A shiny panel complete with the 
  multivariate control chart for 
  qualitative variables, the two ACM 
  charts and the modality distance table. 
  Within the dashboard, arguments such as 
  type I error and dimensionality can be 
  modified. \\ [1ex] 
 \hline
\end{tabular}\caption{Funciones del paquete T2Qv}
\label{tab:functions}
\end{center}
\end{table}

\hypertarget{anuxe1lisis-de-sensibilidad}{%
\section{Análisis de sensibilidad}\label{anuxe1lisis-de-sensibilidad}}

Como se ha mencionado anteriormente, en el gráfico T2Qv, un punto fuera
de control se interpreta como una tabla (\(k_i\)) que incluye una
cantidad o una proporción de variables contaminadas, de tal manera que
la diferencia de los valores de masas de columna, entre de la matriz
\(k_i\) y la matriz consenso, sean significativos según el valor p
obtenido de la distribución \(\chi^2\). En estos casos, se espera que
los puntos en el gráfico T2Qv generalicen el comportamiento de estas
diferencias y superen el límite de control superior (UCL). La ubicación
de este límite de control varía en función del número de dimensiones que
se representen, así, cuando es alto se logra un desempeño óptimo,
mientras que, se introduce inestabilidad y se pierde confiabilidad en
los resultados al disminuir el número de dimensiones de entre las que se
puede representar.

El gráfico de control propuesto es capaz de detectar un punto fuera de
control, aún con un bajo número de variables contaminadas, cuando se
trabaja con un alto número de dimensiones. Se recomienda p - 1, tal que
p es el número total de dimensiones de la matriz inicial (Tabla
\ref{tab:inicial}). Cuando se disminuye el número de dimensiones también
disminuye la altura del límite de control superior (UCL), en
consecuencia, se incrementa el número de puntos fuera de control, aunque
no necesariamente las variables expresen diferencias significativas en
su valores, crece la probabilidad de falsos positivos. Por consiguiente,
la pregunta que surge es hasta cuántas dimensiones se puede disminuir en
el análisis sin perder confiabilidad en el resultado. La importancia de
esta pregunta radica en la necesidad de disponer un gráfico confiable,
que identifique puntos fuera de control aún si se ha aplicado a los
datos una técnica de una reducción de dimensiones, sin caer en casos de
falso positivo.

\begin{figure}[!ht]



\begin{center}\includegraphics[width=0.8\linewidth,]{conjuntosensibilidad} \end{center}

\caption{Curvas de nivel y superficie de respuesta obtenidas con el gráfico T2 Hotelling para variables cualitativas.}

\label{fig:sensibilidad}
\end{figure}

El análisis de sensibilidad utiliza curvas de nivel y superficies de
respuesta (figura \ref{fig:sensibilidad}) para representar el número de
puntos fuera de control, considerando el porcentaje de variables
contaminadas de la \(k_i\) tabla y el número de dimensiones
representadas. Los datos de prueba utilizados en el modelo se registran
en 10 tablas, cada una de ellas incluye 10 variables y cada variable
tiene tres categorías: alto, medio y bajo. La tabla 10 tiene una
distribución diferente de las demás, esta es la tabla contaminada. Se
observa que el modelo es capaz de identificar un punto fuera de control
trabajando con 9 dimensiones (d-1), aún con un porcentaje bajo de
variables contaminadas. Cuando el número de dimensiones disminuye a 8 y
el porcentaje de variables contaminadas es cercano a 100\%, detecta
correctamente 1 punto fuera de control. Se observa además que cuando el
número de dimensiones es menor se pierde estabilidad. En consecuencia,
el análisis de sensibilidad ratifica que el gráfico de control T2Qv
tiene un buen rendimiento cuando trabaja con altas dimensiones.

% %%%%%%%%%%%%%%%%%%%%%%%%%%%%%%%%%%%%%%%%%%
% %% optional
% \supplementary{The following are available online at www.mdpi.com/link, Figure S1: title, Table S1: title, Video S1: title.}
%
% % Only for the journal Methods and Protocols:
% % If you wish to submit a video article, please do so with any other supplementary material.
% % \supplementary{The following are available at www.mdpi.com/link: Figure S1: title, Table S1: title, Video S1: title. A supporting video article is available at doi: link.}

\vspace{6pt}

%%%%%%%%%%%%%%%%%%%%%%%%%%%%%%%%%%%%%%%%%%

%%%%%%%%%%%%%%%%%%%%%%%%%%%%%%%%%%%%%%%%%%

%%%%%%%%%%%%%%%%%%%%%%%%%%%%%%%%%%%%%%%%%%

%%%%%%%%%%%%%%%%%%%%%%%%%%%%%%%%%%%%%%%%%%
%% optional

\input{"appendix.tex"}

%%%%%%%%%%%%%%%%%%%%%%%%%%%%%%%%%%%%%%%%%%
% Citations and References in Supplementary files are permitted provided that they also appear in the reference list here.

%=====================================
% References, variant A: internal bibliography
%=====================================
%\reftitle{References}
%\begin{thebibliography}{999}
% Reference 1
%\bibitem[Author1(year)]{ref-journal}
%Author1, T. The title of the cited article. {\em Journal Abbreviation} {\bf 2008}, {\em 10}, 142--149.
% Reference 2
%\bibitem[Author2(year)]{ref-book}
%Author2, L. The title of the cited contribution. In {\em The Book Title}; Editor1, F., Editor2, A., Eds.; Publishing House: City, Country, 2007; pp. 32--58.
%\end{thebibliography}

% The following MDPI journals use author-date citation: Arts, Econometrics, Economies, Genealogy, Humanities, IJFS, JRFM, Laws, Religions, Risks, Social Sciences. For those journals, please follow the formatting guidelines on http://www.mdpi.com/authors/references
% To cite two works by the same author: \citeauthor{ref-journal-1a} (\citeyear{ref-journal-1a}, \citeyear{ref-journal-1b}). This produces: Whittaker (1967, 1975)
% To cite two works by the same author with specific pages: \citeauthor{ref-journal-3a} (\citeyear{ref-journal-3a}, p. 328; \citeyear{ref-journal-3b}, p.475). This produces: Wong (1999, p. 328; 2000, p. 475)

%=====================================
% References, variant B: external bibliography
%=====================================
\reftitle{References}
\externalbibliography{yes}
\bibliography{mybibfile.bib}

%%%%%%%%%%%%%%%%%%%%%%%%%%%%%%%%%%%%%%%%%%
%% optional

%% for journal Sci
%\reviewreports{\\
%Reviewer 1 comments and authors’ response\\
%Reviewer 2 comments and authors’ response\\
%Reviewer 3 comments and authors’ response
%}

%%%%%%%%%%%%%%%%%%%%%%%%%%%%%%%%%%%%%%%%%%
\end{document}
